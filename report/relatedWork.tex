\section{Related work}
Researchers proposed many \textsc{nmf} algorithms. \citet{lee2001algorithms} first proposes multiplicative update rules (${MUR}$) to minimise Euclidean distance or Kullback-Leibler divergence between the original matrix and its approximation. Although this algorithm is easy to implement and have reasonable convergent rate \citep{lee2001algorithms}, it may fail on seriously corrupted dataset which violates its assumption of Gaussian noise or Poisson noise, respectively \citep{guan2017truncated}.  To improve the robustness of \textsc{nmf}, many methods have been proposed. \citet{lam2008non} proposes ${L_1}$-norm based \textsc{nmf} to model noisy data by a Laplace distribution which is less sensitive to outliers. However, as $L_1$-norm is not differentiable at zero, the optimization procedure is computationally expensive. \citet{kong2011robust} proposed an \textsc{nmf} algorithm using $L_{21}$-norm loss function which is robust to outliers. The updating rules used in $L_{21}$-norm \textsc{nmf}, however, converge slowly because of a continual use of the power method \citep{guan2017truncated}.

Apart from different loss functions, several optimization methods have been proposed to improve the performance of ${NMF}$. After \citet{lee2001algorithms} proposed ${MUR}$, \citet{ lin2007convergence} proposed a modified MUR which guarantee the convergence to a stationary point. This modified ${MUR}$, however, does not improve the convergence rate of traditional ${MUR}$ \citet{guan2012nenmf}. Moreover, as ${MUR}$ is not able to shrink all entries in matrix factors to zero, \citet{berry2007algorithms} proposed a projected nonnegative least square (${PNLS}$) method to overcome this problem. Although, in each nonnegative least square (${NLS}$) subproblem, the least squares solution is directly projected to the negative quadratic, ${PNLS}$ does not guarantee convergence \citet{guan2012nenmf}. In contrast to these gradient-based optimization methods, \citet{kim2008nonnegative} presented an active set method (${AS}$) which divides variables into two sets, free set and active set and update free set in each iteration by solving unconstrained equation. Even though ${AS}$ has good convergence rate, it assumes strictly convex in each ${NLS}$ subproblem \citet{kim2008nonnegative}. 
