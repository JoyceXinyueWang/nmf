\section{Introduction\label{chapter1}}
%Briefly introduce \textsc{nmf}, applications
Non-negative matrix factorization (\textsc{nmf}) is a matrix decomposition technique that approximates a data matrix with non-negative entries with the multiplication of two non-negative matrices
\begin{equation*}
  V \approx WH.
\end{equation*}
In this approximation, matrix~$W$ is the basis and matrix~$H$ is the weight matrix corresponding to the new dictionary~$W$. As \textsc{nmf} only 
%allows additive, non-subtractive combination of matrix factors, it 
is applicable to an extensive range of domain \chenc{I have delete this due to no clarification}. \citet{lee1999learning} suggest that \textsc{nmf} is useful for image processing problems including facial recognition.  

Matrices~$W$ and~$H$ are often referred as the basis images and weights. This is because the observed image $V$ is approximated by a linear combination of~$W$ with positive coefficients~$H$. Due to the additive nature of the algorithm, the dictionary~$W$ are often parts of images. This property also distinguishes \textsc{nmf} from other traditional image processing methods such as principal components analysis (\textsc{pca}). \citet{guillamet2002non} demonstrate that \textsc{nmf} performs better in image classification problems in comparison with \textsc{pca}.

Moreover, \textsc{nmf} is also applicable to text mining such as semantic analysis. Generally, \textsc{nmf} is useful to discover semantic features of an article by counting the frequency of each word, and then approximating the document from a subset of a large array of features \citep{lee1999learning}.

In practice, face images could be easily corrupted during data collection by noise with large magnitude. Corruption may result from lighting environment, facial expression or facial details. An \textsc{nmf} algorithm that is robust to large noise is desired for real-world application. Therefore, the objective of this project is to analyse the robustness of \textsc{nmf} algorithms on corrupted datasets. We implement two \textsc{nmf} algorithms designed by \citet{lee2001algorithms} on real face image datasets, \textsc{orl} dataset and Extended YaleB dataset. We add artificial noises to the face images are contaminated.

The rest of the report is organized as follows. We describe noisy design and two \textsc{NMF} algorithms including Euclidean Distance and Kullback-Leibler Divergence (\textsc{KLD}) in Section 2. Section 3 shows experiment setup and empirical results which demonstrate the robustness of the two NMF algorithms. The conclusions and future work are discussed in Section 5.
